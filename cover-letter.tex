\documentclass{article}
\textheight=25cm

\title{{\small Cover letter concerning the submission of our article:}\\
Early Flow Discard for Packet Scheduling over Wired and Wireless links\\
{\small Submitted IEEE TNMS}
}

\author{Jinbang Chen, Martin Heusse, Guillaume Urvoy-Keller}
\begin{document}


\maketitle


This journal paper submitted to IEEE TNMS follows our two conference papers on Early Flow Discard (EFD) that were accepted for publications at IEEE WOWMOM 2012 and IFIP Networking 2010. The full references of these papers are:

\begin{verbatim}
Jinbang Chen, Martin Heusse, Guillaume Urvoy-Keller:
Analysis of the Early Flow Discard (EFD) discipline in 802.11 
wireless LANs.
WOWMOM 2012: 1-9
\end{verbatim}

\begin{verbatim}
Jinbang Chen, Martin Heusse, Guillaume Urvoy-Keller:
EFD: An Efficient Low-Overhead Scheduler. 
Networking (2) 2011: 150-163
\end{verbatim}

This submission to IEEE TNMS constitutes a novel contribution due to the the following elements:
\begin{itemize}
\item
The EFD algorithm is completely disclosed -- see Algo 1.
\item 
New results concerning the way EFD manages its high and low priority queue, esp. Fig 2, 4, 6 and 7 and the corresponding comments.
\item 
The related work section on the different approaches to detect large flows contains much more material than in our original Networking paper. 
\item 
In the result section  for full and half duplex links, we provided additional results, including confidence intervals (Table 1, Fig 12) with  a discussion on the difficulty to obtain them in the case of high variance distribution of flow sizes.
\item We contrast the size-based scheduling approach with the pure active queue management approach (AQM) proposed in solutions like PIE or CoDel, to alleviate the bufferbloat phenomenon.
\end{itemize}

Overall, the resulting paper that we submit today constitutes in our opinion, a new contribution that sheds an interesting light on recent advances in the domain of size-based scheduling applied to the networking area.

\end{document}